
\section{Situación actual de la generación fotovoltaica}
Según el informe anual de la UNEF\footnote{\textit{UNEF}: Unión Española fotovoltaica} en 2019\cite{unef_2019} la evolución de las energías renovables superó incluso las expectativas mas optimistas alcanzando valores próximos a los 100 GW.

Los aspectos que más han influido en estas cifras han sido, entre otros, la reducción drástica del coste de producción de dichas tecnologías. De hecho, la energía fotovoltaica es ya más barata que la generada por plantas de combustibles fósiles en términos de LCOE\footnote{\textit{LCOE}: Levelized Cost of energy: Medición del coste medio de generación de energía de una planta a lo largo de su vida útil }. Según menciona Bloomber Energy Finance, la fotovoltaica seguirá reduciendo sus costes un 34\% hasta 2030.\\

Sumado a la reducción del coste, el aumento de compraventa de energía a largo plazo - que sigue en alza desde 2018, alcanzando los 14 GW - es otro de los factores influyentes en el crecimiento del sector. Así lo es también la reducción del precio de la energía en las subastas, alcanzando valores tan bajos como 20\$/MWh.\\

Concretamente en Europa, el crecimiento anual de la capacidad solar instalada ha sido de un 23\%, con Alemania como líder sumando otros 2,95 GW respecto a la capacidad del año anterior. En segundo y tercer lugar se encuentran Turquía y los Países Bajos.\\

Bajando un nivel más, nos encontramos con el mercado español, que, según estimaciones del mismo informe de la UNEF, la potencia total instalada experimentó un crecimiento significativo, llegando hasta el valor de 262 MW, sumando la potencia instalada tanto de generación centralizada como la de autoconsumo.\\

Para este año 2020 se estima se estima que la instalación de energía fotovoltaica alcance el umbral de los 20 GW. Si se cumplen las expectativas, la capacidad podría llegar a alcanzar los 200 GW para 2023.\\

Un papel importante lo juegan las autoridades tanto nacionales como a nivel europeo, que apuestan por las fuentes de energías limpias. Para ser exactos, el año 2018 fue uno de los más relevantes en materia de política energética europea desde que se aprobó el tercer paquete de energía en 2009.

De las ocho propuestas que se aprobaron, destaca por su importancia para el sector fotovoltaico la directiva 2018/2001, en la que se recoge el derecho básico al autoconsumo, individual o colectivo, al almacenamiento y sobretodo a la venta de excedentes.\\

En el panorama español, tras varios años de parálisis debida a la compleja situación política de los últimos años, la energía fotovoltaica volvió a recuperar algo de impulso a final del año 2019. Durante el año 2018 y especialmente el 2019, se han incrementado sustancialmente las instalaciones de fotovoltaica, en gran parte gracias a las expectativas generadas por la eliminación del denominado impuesto al sol y la progresiva suspensión de trabas administrativas todas ellas propiciadas por el nuevo marco legal establecido por la administración pública mediante los Reales Decretos 15/2019 y 244/2019.

Según datos proporcionados por la Red Eléctrica Española \footnote{\url{https://www.ree.es/es/datos/publicaciones/series-estadísticas-nacionales}}, la potencia instalada en España ha experimentado un crecimiento de 1,2GW en los últimos 4 años, pasando de 4.6 GW en 2016 a 5.8 GW en 2019.

Se trata del mayor ritmo de crecimiento desde 2008, cuando se instalaron cerca de 2.7 GW de nueva potencia. Es una buena noticia, pero no exenta dificultades debidas sobre todo a la unidireccionalidad de la red eléctrica de nuestro territorio. Sin ir mas lejos, el operador técnico del sistema, REE ha denegado la instalación de 26,3 GW de nueva potencia debido a la imposibilidad de los nudos para gestionar la energía producida por dicha capacidad.

Esto no significa que dicha energía no se vaya a instalar, sino que habrá que esperar a la nueva planificación energética prevista para el periodo 2021-2026 en la que ya están trabajando tanto la REE como las comunidades autónomas para brindar mas oportunidades para los sistemas de conexión a red.\\

En definitiva, como podemos observar, las energías renovables, y en especial la fotovoltaica esta dando mucho que hablar y cada vez es una tema más tratado por el público general y por tanto, es un aspecto a tener en cuenta a la hora de construir nuevas edificaciones o mejorar la eficiencia energética de las existentes.

Surge por tanto la necesidad de aplicaciones y soluciones para la estimación de instalaciones fotovoltaicas que sean intuitivas y fáciles de usar para aprovechar el gran interés que está mostrando el público general.

\section{Soluciones existentes y sus carencias} \label{existing_solutions}

Como ya se ha mencionado en el el apartado \ref{intro_solutions}  existen multitud soluciones para la estimación o simulación de instalaciones fotovoltaicas. Sin embargo, es posible que la aplicación descrita en este proyecto siga teniendo cabida porque ofrece ciertas funciones que, en cierta medida, no existen en las soluciones que se han estudiado.

A continuación se describirán con detalle las soluciones mencionadas anteriormente, junto con sus capacidades y carencias.
\pagebreak

\begin{enumerate}
\item \textbf{PVSyst - Photovoltaic Software} \cite{sota_pvsyst}

El software PVSyst, desarrollado por la empresa suiza con el mismo nombre es quizá el más conocido dentro del ámbito del estudio y la estimación de instalaciones fotovoltaicas. Ofrece una amplia capacidad de personalización de todos los componentes de la instalación.

PVSyst es capaz de calcular con amplio detalle el diseño y dimensionado del sistema, zonas de sombra, envejecimiento del material, almacenamiento de energía, entre otras opciones.\\

PVSyst se diferencia de la aplicación que se desarrolla en este proyecto en algunos puntos importantes como:
	\begin{itemize}
		\item Es de pago, con una licencia anual de aproximadamente \euro{952} mientras que mi solución es gratuita.
		\item Es un programa que se debe instalar en un ordenador de Windows (No funciona en Linux u OSX).
		\item Es poco intuitivo para un usuario con bajos conocimientos de instalaciones fotovoltaicas.
	\end{itemize}
En resumen, PVSyst es un software mucho mas completo y complejo que la solución que yo propongo, y que va enfocada a un público con una base sólida sobre energía fotovoltaica.

\item \textbf{CalculationSolar.com} \cite{sota_calculationsolar}

Como ya se ha mencionado en la introducción, CalculationSolar.com es el primer resultado que aparece en el motor de búsqueda de Google cuando se introduce el término \textit{``calculadora de instalaciones fotovoltaicas''} y por tanto será una de las primeras opciones que un usuario que está interesado en realizar una instalación fotovoltaica considere.

A diferencia de PVSyst, CalculationSolar.com es una solución en la web y gratuita, por lo que la barrera de acceso es mas baja.
Nada mas entrar a la pagina lo primero con lo que nos encontramos es con un formulario sencillo sobre el emplazamiento y la configuración de la instalación. La información que se solicita es sencilla e intuitiva, incluso se ofrece la posibilidad de hacer clic en un mapa interactivo para determinar las coordenadas.

Lo siguiente es introducir la información acerca de las necesidades de potencia, dado que esta calculadora esta enfocada a las instalaciones de autoconsumo sin conexión a red.

Una vez introducidos estos datos, el programa realiza los cálculos y nos ofrece un resultado bastante detallado del campo fotovoltaico, el regulador de carga, la batería y el inversor que mejor encajaría con nuestro requisitos.

La principal diferencia con la aplicación que se desarrolla en este proyecto es requisitos versus limitaciones. Es decir, CalculationSolar.com te permite seleccionar tus requisitos de potencia y te indica el generador que vas a necesita para poder hacer frente a dicha carga. En cambio, SolarCalc indica la potencia y energía que se puede llegar a generar con las limitaciones arquitectónicas impuestas por el usuario. 

\item \textbf{SISIFO} \cite{sota_sisifo}

La solución propuesta por el Grupo de Energías Fotovoltaicas del Instituto de Energía Solar de la Universidad Politécnica de Madrid es también un solución web para el calculo de instalaciones solares tanto de bombeo de agua como conectadas a red.

El sistema de input de información del usuario es muy completo. Paso por paso le permite a éste introducir desde los datos geográficos y meteorológicos hasta hasta los valores de perdidas en los diferentes cableados del sistema. No obstante, los valores por defecto son muy válidos así que incluso un usuario con poco conocimiento sobre energía podría llegar a obtener unos resultados fiables.

Una vez realizada la simulación, la aplicación aporta multitud de resultados detallados tales como Irradiaciones en el plano horizontal e inclinado, temperaturas y energía producida.

Esta aplicación es la más parecida, salvando las distancias, a la que se desarrolla en este proyecto. Sin embargo, es posible que puede llegar a abrumar en cierta medida a un usuario que desea solamente conocer cual es la potencia o energía que puede llegar a producir en su casa, para saber si le va a resultar rentable la inversión.

\item \textbf{PVGIS} \cite{sota_pvgis}

PVGIS es la solución desarrollada por el JRC (Centro de Investigación Conjunta), que forma parte del EU Science Hub. Consiste en una aplicación web que nos permite estimar la energía que puede llegar a producir una instalación fotovoltaica en función de los parámetros introducidos por el usuario.

Tras elegir la localización como primer paso, el programa nos abre la posibilidad de elegir el tipo de instalación, entre conectada a red, con seguimiento, o autónomo.

En este caso, para poder compararlo con la aplicación de este proyecto, se va a realizar la estimación de un sistema conectado a red. La principal diferencia se presenta cuando el programa solicita al usuario la potencia FV pico instalada, así como la tecnología y las perdidas porcentuales del sistema.

Una vez introducidos estos datos, la aplicación nos ofrece los resultados de producción de energía mensual del sistema.

A diferencia de esto, SolarCalc, solicita al usuario el dato de la superficie disponible, para estimar la potencia máxima y por consiguiente, la energía máxima.

\item \textbf{System Advisor Model} \cite{sota_sam}

Esta aplicación desarrollada por el Laboratorio de Energías Renovables del Departamento de Energía de los Estados Unidos. Es un conjunto de soluciones enfocadas a facilitar la toma de decisiones relacionadas con el campo de las Energías Renovables. Entre sus funcionalidades se incluyes programas de cálculo de Fotovoltaica, Termosolar, Eólica, Geotermal o Biomasa.

Es una herramienta muy completa a la vez que compleja, con un enfoque centrado en el apartado económico y la rentabilidad. En cuanto al cálculo y la estimación de una instalación fotovoltaica, ofrece una amplia posibilidad de personalización de cada uno de los campos que afecta a dicho cálculo. Cada uno de los parámetros de la radiación, el módulo, el inversor, las sombras e incluso de la inversión y amortización de la instalación son totalmente ajustables.

Con ésta aplicación sucede como con alguna de las mencionadas anteriormente, que puede llevar a confusión a un usuario medio sin conocimiento relacionados con la fotovoltaica. Estos usuarios son el público objetivo de la aplicación de SolarCalc, desarrollada en este proyecto.

\item \textbf{solaR} \cite{sota_solaR}

Este paquete para R, desarrollado por Oscar Perpiñán, permite llevar a cabo estudios tanto del rendimiento de los sistemas fotovoltaicos como de la radiación solar. Incluye una serie de clases, métodos y funciones para calcular aspectos como la geometría solar, radiación solar incidente sobre un generador, realizar el paso de un generador horizontal a un generador inclinado y orientado y simular el rendimiento de diferentes aplicaciones de la energía fotovoltaica.

A pesar de no ser una aplicación de estimación de instalaciones solares como tal, has sido, junto con el libro \cite{esf_book}, la base de contraste y referencia de los cálculos que se han sido necesario llevar a cabo para poder desarrollar la aplicación de SolarCalc.

\end{enumerate}
