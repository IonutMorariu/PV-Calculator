\documentclass[11pt]{article}
\usepackage[margin=1in]{geometry}
\usepackage{amsfonts,amsmath,amssymb}
\usepackage[none]{hyphenat}
\usepackage[utf8]{inputenc}
\usepackage{fancyhdr}
\usepackage{graphicx}
\graphicspath{{images/}}
\usepackage[nottoc, notlot, notlof]{tocbibind}
\usepackage[spanish]{babel}
\usepackage{hyperref}
\usepackage{eurosym}
\usepackage[T1]{fontenc}
\usepackage{beramono}
\usepackage{listings}
\usepackage{xcolor}

\pagestyle {fancy}
\fancyhead{}
\fancyfoot{}
\fancyhead[L]{\slshape\MakeUppercase{Trabajo de fin de grado}}
\fancyhead[R]{\slshape Ionut Morariu}
\fancyfoot[C]{\thepage}

\parindent 0ex
%\setlength{\parindent}{4em} %Change indent width
%\setlength{\parskip}{1em} %space between paragraphs

\lstdefinelanguage{JavaScript}{
  morekeywords=[1]{break, continue, delete, else, for, function, if, in,
    new, return, this, typeof, var, void, while, with},
  % Literals, primitive types, and reference types.
  morekeywords=[2]{false, null, true, boolean, number, undefined,
    Array, Boolean, Date, Math, Number, String, Object},
  % Built-ins.
  morekeywords=[3]{eval, parseInt, parseFloat, escape, unescape,console},
  sensitive,
  morecomment=[s]{/*}{*/},
  morecomment=[l]//,
  morecomment=[s]{/**}{*/}, % JavaDoc style comments
  morestring=[b]',
  morestring=[b]"
}[keywords, comments, strings]

\lstalias[]{ES6}[ECMAScript2015]{JavaScript}

\lstdefinelanguage[ECMAScript2015]{JavaScript}[]{JavaScript}{
  morekeywords=[1]{await, async, case, catch, class, const, default, do,
    enum, export, extends, finally, from, implements, import, instanceof,
    let, static, super, switch, throw, try},
  morestring=[b]` % Interpolation strings.
}

% Requires package: color.
\definecolor{mediumgray}{rgb}{0.3, 0.4, 0.4}
\definecolor{mediumblue}{rgb}{0.16, 0.5, 0.73}
\definecolor{forestgreen}{rgb}{0.13, 0.55, 0.13}
\definecolor{darkviolet}{rgb}{0.58, 0.0, 0.83}
\definecolor{royalblue}{rgb}{0.25, 0.41, 0.88}
\definecolor{crimson}{rgb}{0.86, 0.8, 0.24}
\definecolor{lightgrey}{rgb}{0.97, 0.97, 0.97}
\definecolor{black}{rgb}{0.05, 0.05, 0.1}
\definecolor{green}{rgb}{0.1529, 0.6823, 0.3764}
\definecolor{red6}{rgb}{0.753, 0.224, 0.169}

\lstdefinestyle{JSES6Base}{
  backgroundcolor=\color{lightgrey},
  basicstyle=\fontsize{10}{12}\selectfont\ttfamily,
  breakatwhitespace=false,
  breaklines=false,
  captionpos=b,
  columns=fullflexible,
  commentstyle=\color{mediumgray}\upshape,
  emph={},
  emphstyle=\color{crimson},
  extendedchars=true,  % requires inputenc
  fontadjust=true,
  frame=single,
  identifierstyle=\color{black},
  keepspaces=true,
  keywordstyle=\color{mediumblue},
  keywordstyle={[2]\color{darkviolet}},
  keywordstyle={[3]\color{red}},
  numbers=left,
  numbersep=10pt,
  numberstyle=\tiny\color{black},
  rulecolor=\color{black},
  showlines=true,
  showspaces=false,
  showstringspaces=false,
  showtabs=false,
  stringstyle=\color{green},
  tabsize=2,
  title=\lstname,
  upquote=true  % requires textcomp
}

\lstdefinestyle{JavaScript}{
  language=JavaScript,
  style=JSES6Base
}
\lstdefinestyle{ES6}{
  language=ES6,
  style=JSES6Base
}

\begin{document}

\begin{titlepage}



\begin{center}
	\includegraphics[scale=1]{cabecera}\\
	\vspace*{1cm}
	\Large{\textbf{\MakeUppercase{Universidad Pólitécnica de Madrid}}}\\[3mm]
	\Large{{\MakeUppercase{Escuela técnica de ingeniería y diseño industrial}}}\\[3mm]
	\Large {Grado en Ingeneriería Eléctrica}\\
	\vfill
	\line(1,0){400}\\
	\Large{{\MakeUppercase{Trabajo de fin de grado}}}\\
	\Huge{\textbf{Aplicación web para la estimación del coste de una instalación solar conectada a red}}\\[5mm]
	\Large{Autor: Ionut Cristian Morariu}\\
	\line(1,0){400}\\
	\vfill
\end{center}
\begin{flushright}
\Large {Tutor: Óscar Perpiñán Lamigueiro}\\[3mm]
\Large{Departamento de Ingeniería Eléctrica,\\ Electrónica, Automática y Física aplicada}\\[10mm]
Madrid, \today
\end{flushright}

\end{titlepage}

\renewcommand{\baselinestretch}{1.5} %line spacing
\renewcommand{\labelitemi}{\textbullet}

\tableofcontents
\thispagestyle{empty}
\clearpage

\setcounter{page}{1}


\section{Introducción}
\subsection{Objetivos}
El objetivo de este proyecto es el desarrollo de una aplicación web de fácil acceso para la estimación del coste de una instalación fotovoltaica doméstica.\\

El usuario introducirá unos serie de datos acerca de su emplazamiento y la aplicación hará los cálculos necesarios para ofrecer una aproximación lo más concreta la instalación.\\

La aplicación también ofrecerá otros datos de posible interés para el usuario como: el número de paneles que se pueden instalar, la potencia de dichos paneles y la potencia del inversor.\\

La idea de esta aplicación surge de una conversación que tuve con un conocido cuando estaba planificando la construcción de su nueva vivienda, en la cual quería realizar una instalación fotovoltaica para reducir el gasto en la factura de electricidad.\\

En su búsqueda no encontró ningún servicio que fuera lo suficientemente sencillo de entender para una persona sin ningún tipo de conocimiento previo acerca de la generación fotovoltaica y ademas tuviese la posibilidad de adaptar los cálculos a un emplazamiento concreto.\\


Los objetivos detallados de esta aplicación son los siguientes:
\begin{itemize}
\item Diseñar una interfaz de usuario amigable y sencilla de usar para que la pueda utilizar un gran número de personas sin necesidad de conocimientos sobre energía fotovoltaica.

\item Obtención de los datos de radiación en el emplazamiento indicado por el usuario mediante el uso de API\footnote{\textit{Application Programming Interface}: conjunto de funciones y procedimientos que ofrece la posibilidad de un software a interaccionar con otro.} externas.
\item Realizar todos los cálculos necesarios para ofrecer una estimación competente de los siguientes datos:

\begin{itemize}

\item Número de paneles que se pueden instalar.
\item Potencia máxima a instalar.
\item Potencia del inversor.
\item Energía eléctrica producida en un año.

\end{itemize}
\end{itemize}
\newpage

\subsection{Aspectos técnicos}

\subsubsection{Backend}

Para el Backend \footnote{\textit{Backend:} Término utilizado para referirse a la parte de una pagina web encargada de tratar las peticiones del usuario en la página.} se ha empleado una tecnología basada en Javascript llamada NodeJS \footnote{\textit{NodeJS:} Entorno de ejecución basado en el motor de Chrome llamado V8. \url{https://nodejs.org/en/} }, con la ayuda de las librerias ExpressJS\footnote{\textit{ExpressJS:} Framework web para el entorno de NodeJS. \url{https://expressjs.com/es/}} y Mongoose \footnote{\textit{Mongoose:} ORM para las BBDD MongoDB \url{https://mongoosejs.com/}}.   \\
La base de datos que se ha utilizado para almacenar los datos necesarios ha sido MongoDB.

En el servidor se realizan varias tareas relacionadas con los cálculos necesarios. Algunas de estas tareas son:
\begin{itemize}
\item Obtención de los datos de radiación global media en el plano horizontal para el emplazamiento indicado
\item Proceso completo de cálculo para pasar de la radiación en el plano horizontal al plano inclinado y orientado según los datos introducidos por el usuario
\item Proceso de obtención de los datos relacionados con el perfil horario de temperatura en el emplazamiento indicado
\item Gestión de las diferentes rutas que constituyen la API.
\end{itemize}

\subsubsection{Frontend}
 Para el Frontend \footnote{\textit{Frontend:} Término utilizado para referirse a la parte de visual de una web, con la que interactúa el usuario.} de la página se han utilizado las tres tecnologías necesarias para poder desarrollar una pagina web: HTML5, CSS3, Javascript.\\
 
La unica libreria utilizada ha sido Bulma \footnote{\textit{Bulma:} Libreria de componentes CSS. \url{https://bulma.io}} para ahorrar tiempo a la hora de darle un aspecto visual agradable a la página.\\

Esta parte de la página es la encargada de recoger los datos del usuario  y enviarlos al servidor para que se realicen los cálculos. Una vez realizados dichos cálculos, la página mostrará la información relevante al usuario, junto con algunos unos gráficos adicionales.\\

Tanto el backend como el frontend están almacenados en un servidor de Linux ofrecido por la empresa DigitalOcean.\\

Todas las tareas mencionadas tanto en la parte de Backend como en la parte de Frontend se describirán en detalle en la sección \ref{sec:theory}, junto con todos los cálculos en los que se ha basado. 

\newpage

\section{Estado del arte}



\subsection{Situación actual del mercado}

\subsection{Soluciones existentes y sus carencias}


\section{Parte teórica y desarrollo del código}
\label{sec:theory}

\section{Ejemplo práctico de aplicación}

\section{Conclusión}


\pagebreak

\begin{thebibliography}{99}
 
\end{thebibliography}

\end{document}
